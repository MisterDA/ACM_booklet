\documentclass[12pt]{article}
\usepackage[a5paper,margin=2cm]{geometry}
\usepackage[french]{babel}
\usepackage{amssymb,amsthm,amsmath}
\usepackage{xltxtra}
\usepackage{stmaryrd}
\usepackage{graphicx}
\usepackage{listings}
\usepackage{color}
\lstset{
        extendedchars=false,
        showstringspaces=false,
        escapeinside=``,
        keywordstyle=\color{blue},
        commentstyle=\color[rgb]{0.133,0.545,0.133},
        columns=flexible,
        language=C++,
        tabsize=4,
        basicstyle=\ttfamily,
        numbers=left,
        frame=lines
}

\begin{document}
\tableofcontents\vspace{0.5cm}

\section{Graphes}
\subsection{Dijkstra}
{\scriptsize\lstinputlisting{code/dijkstra.cc}}

\subsection{Floyd-Warshall}
{\scriptsize\lstinputlisting{code/floydwarshall.cc}}

\subsection{Bellman-Ford}
{\scriptsize\lstinputlisting{code/bellmanford.cc}}

\subsection{Kruskal-Prim}
{\scriptsize\lstinputlisting{code/kruskalprim.cc}}

\subsection{Parcours Eulerien}
{\scriptsize\lstinputlisting{code/eulertour.cc}}

\subsection{SCC}
{\scriptsize\lstinputlisting{code/scc.cc}}

\subsection{Max Flow}
{\scriptsize\lstinputlisting{code/maxflow.cc}}

\subsection{MCMF}
{\scriptsize\lstinputlisting{code/mcmf.cc}}

\subsection{Min Cut}
{\scriptsize\lstinputlisting{code/mincut.cc}}

\subsection{Couplage Maximal (biparti)}
{\scriptsize\lstinputlisting{code/bipartitematch.cc}}

\subsection{Algorithme Hongrois}
{\scriptsize\lstinputlisting{code/hungarian.cc}}

\subsection{Ford-Fulkerson}
{\scriptsize\lstinputlisting{code/fordfulkerson.cc}}

\subsection{BFS}
{\scriptsize\lstinputlisting{code/bfs.cc}}

\subsection{DFS}
{\scriptsize\lstinputlisting{code/dfs.cc}}


\section{Strings}
\subsection{KMP}
{\scriptsize\lstinputlisting{code/kmp.cc}}

\subsection{Boyer-Moore}
{\scriptsize\lstinputlisting{code/boyermoore.cc}}

\subsection{Hashing}
{\scriptsize\lstinputlisting{code/hash.cc}}

\subsection{Rabin-Karp}
{\scriptsize\lstinputlisting{code/rabinkarp.cc}}

\subsection{Algorithme Z}
{\scriptsize\lstinputlisting{code/zalg.cc}}

\subsection{Manacher (sous-palindrome le plus long)}
{\scriptsize\lstinputlisting{code/manacher.cc}}


\section{Structures}
\subsection{Tableaux de Suffixes, LCP}
{\scriptsize\lstinputlisting{code/suffixarray.cc}}

\subsection{Tries}
{\scriptsize\lstinputlisting{code/trie.cc}}

\subsection{Arbres de segments}
{\scriptsize\lstinputlisting{code/segmenttree.cc}}

\subsection{Treaps}
{\scriptsize\lstinputlisting{code/treap.cc}}

\subsection{Arbres binaires indexés}
{\scriptsize\lstinputlisting{code/bit.cc}}


\section{Géométrie}
\subsection{Enveloppe convexe}
{\scriptsize\lstinputlisting{code/convexhull.cc}}

\subsection{Notes}
\begin{itemize}
  \item Sphere surface : $4\pi r^2$
  \item Cone surface : $\pi r^2 + \pi r h$
  \item Cone volume : $\frac{1}{3} \pi r^2 h$
  \item Cap surface : $2 \pi r h$
  \item Cap volume : $\frac{\pi h}{6}(3 a^2 + h^2)$
  \item Cap relation : $r = \frac{a^2 + h^2}{2h}$
\end{itemize}



\section{Arithmétique et Algèbre}
\subsection{Exponentation modulaire rapide}
{\scriptsize\lstinputlisting{code/fastexpmod.cc}}

\subsection{DFT NTT, convolution}
{\scriptsize\lstinputlisting{code/dftntt.cc}}

\subsection{Coefficients Binomiaux}
{\scriptsize\lstinputlisting{code/binomials.cc}}

\subsection{Equations diophantiennes linéaires}
{\scriptsize\lstinputlisting{code/diophantine.cc}}

\subsection{Prochaine permutation (lexicographique)}
{\scriptsize\lstinputlisting{code/nextperm.cc}}

\subsection{Enumeration de permutations}
{\scriptsize\lstinputlisting{code/enumperm.cc}}

\subsection{Systèmes linéaires}
{\scriptsize\lstinputlisting{code/linear.cc}}

\subsection{Déterminant}
{\scriptsize\lstinputlisting{code/determinant.cc}}

\subsection{Inversion de matrice}
{\scriptsize\lstinputlisting{code/matinverse.cc}}

\subsection{Multiplication chaînée}
{\scriptsize\lstinputlisting{code/matchain.cc}}

\subsection{Simplexe}
{\scriptsize\lstinputlisting{code/simplex.cc}}
 
\subsection{Union Find}
{\scriptsize\lstinputlisting{code/unionfind.cc}}

\subsection{Crible}
{\scriptsize\lstinputlisting{code/sieve.cc}}

\subsection{Algorithme d'Euclide Etendu}
{\scriptsize\lstinputlisting{code/exteuclide.cc}}

\subsection{Kadane (sous tableau de somme max)}
{\scriptsize\lstinputlisting{code/kadane.cc}}



\subsection{Recherches (dichotomie, trichotomie)}
 


\subsection{Notes}
 



\section{Bonus}
\subsection{Template}
{\scriptsize\lstinputlisting{code/template.cc}}

\subsection{Notes : STL}
 


\end{document}

